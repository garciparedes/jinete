% !TEX root = ../../document.tex

\documentclass{subfiles}

\begin{document}

  \chapter{Introducción}
  \label{chap:intro}

    \section{Contexto}
    \label{sec:introduction_context}

      \paragraph{}
      Este documento se enmarca sobre el contexto de un \emph{trabajo de fin de grado}. En concreto, se corresponde con la titulación del \emph{Grado en Estadística} impartido en la \emph{Universidad de Valladolid}. El peso de este proyecto en el plan de estudios de la titulación, se corresponde con \emph{6 créditos ECTS}, lo cual implica unas 180 horas de trabajo (sin embargo, esto es una medida de referencia). Debido al carácter científico de la titulación, gran parte del documento será dedicada a la definición y razonamiento acerca de las bases teóricas que sustentan los métodos analizados.

      \paragraph{}
      A pesar de ello, también se pretende dedicar gran peso del trabajo a las estrategias de cómputo que permiten llevar a cabo los métodos analizados de manera eficiente. Esto se justifica debido al elevado coste computacional de algunos de los métodos que se analizarán, especialmente cuando se trabaja con conjuntos de observaciones cuyo tamaño muestral es elevado.

    \section{Motivación}
    \label{sec:introduction_motivation}

      \paragraph{}
      [TODO]

    \section{Ideas Iniciales}
    \label{sec:introduction_initial_ideas}

      \paragraph{}
      [TODO]

    \section{Objetivos}
    \label{sec:introduction_goals}

      \paragraph{}
      [TODO]

\end{document}
